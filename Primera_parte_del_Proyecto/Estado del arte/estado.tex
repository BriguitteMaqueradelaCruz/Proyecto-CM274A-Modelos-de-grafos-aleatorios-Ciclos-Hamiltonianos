\documentclass[11pt,a4paper]{article}
\usepackage[utf8]{inputenc}
\usepackage{amsmath}
\usepackage[spanish]{babel}
\usepackage{amsfonts}
\usepackage{amssymb}
\usepackage{multicol}
\usepackage{graphicx}
\usepackage[pdftex]{color}
\usepackage[left=2.00cm, right=2.00cm, top =2.00cm, bottom =2.00cm]{geometry}
\begin{document}
\begin{center}
II. ESTADO DEL ARTE
\end{center}
El problema respecto a los grafos, espec{\'i}ficamente los ciclos hamiltonianos, vienen siendo una adversidad hist{\'o}rica.\\
En \textbf{1736} Leonhard Euler en uno de sus viajes en Koñigsberg en la costa del Mar B{\'a}ltico, en la Prusia oriental (Rusia) hab{\'i}an siete puentes distribuidos donde plane{\'o} un paseo de manera que saliendo de casa cruce los siete puentes una sola vez cada uno antes de regresar a casa haciendo referencia a los caminos hamiltonianos.\\
En \textbf{1805} Roman Hamilton se propuso a viajar a 20 ciudades del mundo, representadas como los v{\'e}rtices de un dodecaedro regular, siguiendo las aristas del dodecaedro.\\
En \textbf{1824} Kirchoff se sirvi{\'o} de la Teor{\'i}a de Grafos para enunciar las leyes que permiten el c{\'a}lculo de voltajes y circuitos el{\'e}ctricos.
\end{document}