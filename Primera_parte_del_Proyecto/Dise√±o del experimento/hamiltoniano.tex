\documentclass[14pt]{article}
\usepackage[latin1]{inputenc}
%\title{DISE�O DEL EXPERIMENTO}
\usepackage[margin=3cm]{geometry}
\begin{document}
%\begin{center}
%\textbf{\LARGE{\underline{DISE�O DEL EXPERIMENTO}}}
%\end{center}
\section{DISE�O DEL EXPERIMENTO}
\subsection{�Es G hamiltoniano?}
\begin{itemize}
	\item No existe ning�n m�todo general v�lido aplicable a todos los grafos para determinar si es o no hamiltoniano.
	\item	El m�todo que trataremos a continuaci�n es v�lido, en t�rminos generales, si el grafo tiene v�rtices de grado dos y no tiene un gran n�mero de aristas (aunque la aplicabilidad o no del m�todo depende siempre del grafo en concreto).
\end{itemize}

\subsection{�En qu� tipo de m�todo se trata?}
\begin{itemize}
	\item El m�todo es constructivista, buscando no s�lo la existencia sino el ciclo hamiltoniano, caso de que exista.
\end{itemize}

\subsection{�En qu� se apoya?}
\begin{itemize}
	\item El m�todo se apoya en el hecho de que existe un ciclo hamiltoniano, �ste debe contener exactamente dos de las aristas de cada uno de los v�rtices (por definici�n de ciclo).
\end{itemize}

\subsection{Estrategia}
\begin{itemize}
	\item	Dado un grafo no dirigido G, con $|v|>2$, suponemos que tiene ciclo hamiltoniano e intentaremos construirlo a partir de cuatro reglas.
	\item	Si las reglas 1 o 4 no se cumplen, el grafo no ser� hamiltoniano.
	\item En caso contrario habremos obtenido, tras un n�mero determinado de pasos, un ciclo hamiltoniano en G.
\end{itemize}

\subsection{Reglas}
	Suponemos que existe un ciclo hamiltoniano C en G.
\begin{description}
	\item[Regla 1:]  Si existe ciclo hamiltoniano en G entonces todos los v�rtices tienen grado mayor o igual que dos.
	\item[Regla 2:] Sea v un v�rtice de grado 2. Entonces las dos aristas incidentes en v deben pertenecer al ciclo C.
	\item[Regla 3:] Si v es un v�rtice de grado mayor que 2, y ya hemos incorporado al ciclo C que estamos reconstruyendo dos de sus aristas, el resto de aristas incidentes en v deben ser desechadas.
	\item[Regla 4:] Si el grafo es realmente hamiltoniano, con la construcci�n ?obligada? que estamos realizando no podemos encontrar un ciclo que contenga un n�mero de v�rtices menor que $|v|$.
\end{description}

\section{IMPLEMENTACI�N EN R}

\subsection{Funciones y t�cnicas a usar}

\begin{description}
	\item[Plot:] La funci�n plot es una funci�n gen�rica para la representaci�n gr�fica de objetos en R. Los gr�ficos m�s sencillos que permite generar esta funci�n son nubes de puntos (x, y).
	
	\item[Grafos con igraph:] El paquete para Igraph, necesita que se le presente los datos de la matriz de adyacencia por parejas. Es decir, una matriz de doble entrada convencional (tambi�n llamada sociomatriz, tabla de confundido o tabla de concordancia) ha de pasarse al formato de igraph.
	
	\item[Archivos CSV:] Archivo de texto que contiene una serie de valores separados por comas. Los valores pueden ser cualquier cosa, desde n�meros de un presupuesto de una hoja de c�lculo, hasta nombres y descripciones de una lista de clientes de un negocio.
\end{description}

\end{document}